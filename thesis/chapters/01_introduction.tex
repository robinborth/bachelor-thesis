\chapter{Introduction}\label{ch:introduction}


\section{Motivation}\label{sec:motivation}
A \ac{UAV} is a cyber-physical system.
In contrast to classical systems, the faults in cyber-physical systems are often caused by their interaction with their physical environment and the social context.
Thus, when \ac{UAV}s crash, it is difficult to identify the root cause of the failure.
If we cannot identify the cause, we cannot fix the issue and, therefore, not prevent it in the future.
Experts manually create causal graphs with semi-automatic procedures \cite{ibrahim2019practical} to trace back the events that led to the final failure.
Nevertheless, its formation costs a lot of time and effort, as it is not fully automized yet.
Therefore, these graphs are strongly limited in their knowledge scope and sometimes not detailed enough.
However, several sources provide a substantially wealthier knowledge base, such as online documentation and forums.
For instance, ArduPilot\footnote{https://ardupilot.org/} is an open/source autopilot system that provides comprehensive online documentation and has an active community of experienced users and developers.
If we could use all that valuable information at hand and create an automatic causal graph, we would have a powerful tool to resolve innumerable issues with \ac{UAV}s efficiently.
To build such a graph, we would have to collect cause/effect pairs informally reported by users and experts online in natural language.
For example, an experienced user argues that \qq{a \ac{GPS} fault causes a crash} in a forum post.
Such information can be encoded as a pair of graph nodes with a direct edge between them, with \qq{GPS fault} as the cause and \qq{crash} as the effect.


\section{Problem And Research Questions}\label{sec:problem}
As mentioned above, domain experts create causal graphs that allow backtracking from effects to their causes.
However, if we automized this creation procedure and utilized the knowledge online, we could create an enhanced graph that is more powerful and rich in knowledge.
Frameworks allow building causal graphs given the \ac{CEP}s.
Therefore, our main task is to extract those pairs.
Yet, there are several problems
\begin{enumerate}
    \item The knowledge is scattered across several online platforms.
    \item The \ac{CEP}s are embedded in natural language, hindering us from extracting them directly.
    \item The same information may be expressed differently, e.g., with synonyms, leading to imprecision in our graph.
\end{enumerate}
Some methods exist to resolve these issues, for instance, web scraping to scrape all information sources or \ac{NLP} to extract formal information from natural language.
However, there is no procedure combining these methods and adapting them for our specific domain to build an extensive causal graph.
In this thesis, we want to fill that gap.

Furthermore, we want to gain insight into the resulting causal graph and validate our graph against the already-available sanitized dataset of \ac{UAV} flight logs.
In particular, we aim to answer the following research questions:
\begin{enumerate}
    \item How diverse are the mentioned cause and effect events in the online resources?
    \item How consistent are the causal relationships in the online resources?
    \item How detectable are such cause and effect events in the data set?
    \item How accurate are the mentioned causal relationships considering the historical data?
\end{enumerate}


\section{Objectives}\label{sec:objectives}
To complete our ultimate goal of building the causal graph, we define the following objectives: We want to develop a system that automatically extracts the knowledge from online resources and outputs its causal graph.
Additionally, we want to measure our procedure’s performance, evaluate our graph’s quality, and answer the above research questions.


\section{Solution}\label{sec:solution}
The first step is to gather textual data from domain experts.
We decided to use three different information resources of the ArduPilot community: the online documentation\footnote{\url{https://ardupilot.org/ardupilot/}}, discussion forum\footnote{\url{https://discuss.ardupilot.org/}}, and Discord channel \footnote{\url{https://discord.com/channels/674039678562861068/674039678982422579}}.
We utilize state-of-the-art web scraping to achieve that.
Then, we use sentence segmentation to extract all the sentences that users and developers stated.
To deal with \qq{dirty} sentences, we utilize regular expressions to clean them and apply grammar rules to filter incomplete sentences.
Next, we extract \ac{CEP}s with the following three steps: 1) We pass the sentence to a pre-trained \ac{NLP} model, which adds linguistic features and tokenizes the sentence.
2) We utilize dependency patterns using Semgrex operators on a dependency tree to identify the root of the cause and effect.
3) We extend each root to one or multiple phrases by following the sentence's dependency relations between the tokens.
Afterward, we build a synonym dictionary to unify \ac{CEP}s with the same meaning to ensure conciseness.
Now, we can assemble the resulting \ac{CEP}s into a holistic \ac{WDCG} that can identify different root causes of events.
Lastly, we define several quality measurements that evaluate the graph’s performance.


\section{Contributions}\label{sec:contributions}
During this thesis, we contributed the following to \ac{UAV} research:
\begin{itemize}
    \item The methodology to build an automized pipeline that outputs an extensive causal graph from various resources.
    \item An extracted, processed, and sanitized knowledge base from the community for further research with over 724.922 sentences.
    \item A phrase extraction algorithm is ably extracting multiple atomic \ac{CEP}s from one explicit causal mention.
    \item The \ac{APD} consists of 69 labeled sentences that can have multiple causes and effects.
\end{itemize}


\section{Summary Of Results}\label{sec:results}
During the development of our system, we were able to gain different insights about \ac{UAV} systems from the ArduPilot community and our resulting causal graph.
For one, we learned that only a few causal relationships we found are reflected in the historical data.
However, those we found in the flight logs dataset are mostly correct.
Therefore, we assume that the lack of coverage is not due to incorrectness but to the limitations of the historical data set.
On the contrary, it underlines the wealth of knowledge found in online sources.
Another insight we gained is that the reported \ac{CEP}s are highly diverse, i.e., that most people reported different \ac{CEP}s.
Furthermore, we showed consistency in our relations, as we barely found contradictions.
Due to the nature of the informal natural language used in forums and chats, the data was noisy.
For instance, users referring to a previous post with \qq{this},\qq{that}, or \qq{it} lead to useless \ac{CEP}s such as \qq{this causes that}.
Thus, we filtered such pairs to ensure meaningfulness in our graph.
Overall, we built a system that scraped 724.922 sentences, extracted 3301 \ac{CEP}s, and constructed an extensive \ac{WDCG}.
Our \ac{CEP} extraction algorithm achieved a performance of \qq{0.632} precision at a recall of \qq{0.662} with the resulting f1-score of \qq{0.647} on \ac{APD}.
Furthermore, we achieved a precision of \qq{0.9655} at a recall at \qq{0.6437} and \qq{0.7724} f1 score at \ac{NATO-SFA}, which is comparable with similar approaches.


\section{Outline}\label{sec:outline}
This thesis will guide one through implementing a pipeline for causal graph construction.
In \autoref{ch:related-work}, the Related Work, we analyze the existing approaches for web scraping, \ac{CEP} extraction, and graph building.
Based on that, we choose those appropriate for our use case.
Next, in \autoref{ch:pipeline-overview}, the Pipeline Overview, we show the big picture of the necessary steps to generate a causal graph automatically.
The first pipeline module is the Data Collection Module (see chapter \autoref{ch:data-collection}), where we analyze the different sources and adapt our web scraping approaches accordingly.
Next, in \autoref{ch:cause-effect-extraction}, the Cause-Effect-Extraction module, we present the patterns and rules to extract \ac{CEP}s.
In \autoref{ch:causal-graph}, the Causal Graph module, we demonstrate how we unify the \ac{CEP}s.
We also provide a methodology to map the flight logs dataset into a \ac{WDCG} .
In the last pipeline module, the Evaluation (see \autoref{ch:evaluation-methods}), we formalize our performance measurement of the \ac{CEP} extraction algorithm.
This chapter also provides the formulas we used to evaluate the \ac{WDCG} .
Then, in \autoref{ch:results}, the Results, we showcase our results and the knowledge we gained from the causal graph.
Finally, in the last \autoref{ch:conclusions}, the Conclusion, we wrap up the thesis, show the limitations of our approach, and suggest future work possibilities.
