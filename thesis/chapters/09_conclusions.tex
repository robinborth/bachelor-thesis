\chapter{Conclusions}\label{ch:conclusions}
In this thesis, we extracted the vast wealth of knowledge online and made it accessible through a causal graph.
We built a system that scrapes the web from several resources, formalizes the knowledge with \ac{NLP}, extracts \ac{CEP}, and constructs an extensive causal graph for the \ac{UAV} domain.
The methods and technologies we used for \ac{CEP} extraction are extendable, allowing others to contribute to the project efficiently and adaptable, allowing us to apply the procedure for other domains.
Many relations in our graph consist of problems and crashes of a \ac{UAV} helping to find root causes of these failures more efficiently.
In addition, with the significant consistency in our results, we built a causal graph that helps to resolve and backtrack an innumerable number of problems regarding \ac{UAV}s.


\section{Limitations}\label{sec:limitations}

Our system can build a causal graph.
However, there are still some limitations due to the time constraint of the thesis.
The current web scraping implementation for the discussion forum uses only one node at the Selenium Grid.
The \ac{CEP} extraction algorithm currently consists only of three patterns.
Therefore, some pairs might be left out.
Another problem is that we cannot retrace \ac{CEP}s across sentences or even posts.
Furthermore, due to using only explicitly mentioned \ac{CEP}s, we might miss potentially knowledge from the users.
To build the causal graph, we used a dictionary of synonyms as a generalization step, which is promising.
However, we do not merge similar nodes, e.g., \qq{crash} and \qq{big crash}, which would lead to a more dense and compact graph.
Also, the Ardupilot dataset which we provide consists of only 69 sentences.


\section{Future Work}\label{sec:future-work}

In this last section, we suggest future work possibilities to optimize our procedure further.
To improve the data collection module, we could apply a more general scraping technique based on web crawlers that gather data from unseen domains.
We could then transfer the pipeline to other domains that could benefit from a causal graph, such as medical or financial domains.
We could replace the pre-trained model based on word embeddings with a transformer-based model, which we train on the whole corpus of sentences.
This could lead to more accurate dependencies between tokens, improving the results for our dependency patterns.
As we mentioned in \autoref{ch:related-work} several machine learning solutions are promising to extract \ac{CEP}s.
We could extend the Ardupilot dataset and use it for a machine learning approach that requires such a labeled dataset to improve the results.
The last suggestion would be to improve the visualization tool that allows only to discover the causal graph by looking at the nodes.
One potential improvement could be to use an advanced text similarity measurement to find \ac{CEP}s of interests.
